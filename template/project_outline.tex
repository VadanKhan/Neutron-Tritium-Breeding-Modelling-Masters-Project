% DOC SETUP

% revtex class:
% \documentclass[%
%  onecolumn,
%  nofootinbib,
%  amsmath,amssymb,
%  aps,
%  12pt,
%  nolongbibliography,
% ]{revtex4-2}

% base article class:
\documentclass[12pt]{article}
% revtex loaded these for you, so now we add them explicitly
\usepackage{amsmath,amssymb}

% base article class:
% % Add the biblatex package
\usepackage[
  backend=biber,
  style=vancouver,  % Use the Vancouver style
  natbib=true,
  maxbibnames=6,    % Truncate if > 6 authors
  minbibnames=6     % <-- THE FIX: When truncating, print 6
]{biblatex}
% % Tell biblatex where to find your references
\addbibresource{thesis.bib} % <-- Make sure to include the .bib extension
\renewcommand*{\bibfont}{\small\linespread{1.0}\selectfont}
\usepackage{graphicx}% Include figure files
\usepackage{dcolumn}% Align table columns on decimal point
\usepackage{bm}% bold math
\usepackage{hyperref}% add hypertext capabilities
\usepackage[title]{appendix}
\usepackage[a4paper, margin=2cm]{geometry}
\usepackage{titlesec}
% --- Custom Title Formatting and Spacing ---

% 1. Set numbering depth to include \paragraph (level 4)
\setcounter{secnumdepth}{4}

% 2. Define \section (Large, BOLD, UPPERCASE, Ragged-Right)
% We MUST use the un-starred version to use \MakeUppercase
\titleformat{\section}
  {\normalfont\large\bfseries\raggedright} % Format: Large, bold, no right-justification
  {\thesection.}                          % Label
  {1em}                                   % Space between label and title
  {\MakeUppercase}                        % Command to apply to the title text
\titlespacing*{\section}
  {0pt}{3.5ex plus 1ex minus .2ex}{2.3ex plus .2ex}

% 3. Define \subsection (Normal size, BOLD)
% This is now \normalsize and \bfseries (bold) as you requested
\titleformat{\subsection}
  {\normalfont\normalsize\bfseries} % <-- CHANGED: normalsize, bold
  {\thesubsection.}
  {1em}
  {}
\titlespacing*{\subsection}
  {0pt}{3.25ex plus 1ex minus .2ex}{1.5ex plus .2ex}
% 4. Define \subsubsection (Normal size, bold-italic)
% This is now \bfseries\itshape (bold-italic) to distinguish it.
\titleformat{\subsubsection}
  {\normalfont\normalsize\bfseries\itshape} % <-- CHANGED TO BOLD-ITALIC
  {\thesubsubsection.}
  {1em}
  {}
\titlespacing*{\subsubsection}
  {0pt}{2.5ex plus 0.5ex minus .2ex}{1.2ex plus .2ex}

% 5. Define \paragraph (Normal size, italic)
% This is now \itshape (italic) as requested.
\titleformat{\paragraph}
  {\normalfont\normalsize\itshape} % <-- CHANGED TO ITALIC
  {\theparagraph.}
  {1em}
  {}
\titlespacing*{\paragraph}
  {0pt}{1.5ex plus 0.2ex}{1ex}
% --- End of Custom Title Formatting ---
\usepackage{mhchem}
\usepackage{booktabs} % <-- Add this to your preamble if not already there


% revtex class:
% \usepackage{etoolbox} % Make sure this is in your preamble
% % Add commands before the environment starts
% \BeforeBeginEnvironment{thebibliography}{%
%%   \begingroup                 % Start a group to keep changes local
%   \linespread{1.0}\selectfont % Set line spacing to single
%   \small                      % Set font size to \small
% }

% % Add commands after the environment ends
% \AfterEndEnvironment{thebibliography}{%
%   \endgroup % End the local changes
% }




%\usepackage[showframe,%Uncomment any one of the following lines to test
%%scale=0.7, marginratio={1:1, 2:3}, ignoreall,% default settings
%%text={7in,10in},centering,
%%margin=1.5in,
%%total={6.5in,8.75in}, top=1.2in, left=0.9in, includefoot,
%%height=10in,a5paper,hmargin={3cm,0.8in},
%]{geometry}


%%%  Some modifications to reduce the space used by RevTex for sections, and use traditional numbering
% % Usual (decimal) numbering
% \renewcommand{\thesection}{\arabic{section}}
% \renewcommand{\thesubsection}{\thesection.\arabic{subsection}}
% \renewcommand{\thesubsubsection}{\thesubsection.\arabic{subsubsection}}

% % Fix references
% \makeatletter
% \renewcommand{\p@subsection}{}
% \renewcommand{\p@subsubsection}{}
% \makeatother

%% Use this if you want times new roman which is slightly more compact.
%\usepackage{times}
\setlength{\parindent}{10pt}
\linespread{1.2}


% \titlespacing\section{0pt}{12pt plus 4pt minus 2pt}{6pt plus 2pt minus 2pt}
% \titlespacing\subsection{0pt}{5pt plus 4pt minus 2pt}{5pt plus 2pt minus 2pt}
% \titlespacing\subsubsection{0pt}{4pt plus 4pt minus 2pt}{4pt plus 2pt minus 2pt}

% \titleformat*{\section}{\raggedright\bf\MakeUppercase}
% \titleformat*{\subsection}{\itshape}


%% READ BEFORE USE IF HAVING CITATION TROUBLES The natlib package (used for citaitons here) uses
% \citep{} to change the citations to have square brackets over the author and year. Just using
% \cite{} will only bracket the year. For more harvard citation options visit this url:
% https://library-guides.imperial.ac.uk/c.php?g=719784&p=5224027




% DOC BEGIN

\begin{document}

\section{Introduction}

\section{Theory}
\subsection{Fundamentals of Nuclear Fusion}
\subsection{Neutron Transport and Interactions}
\subsection{Tritium Breeding and Neutron Multiplication}

\section{Literature Review}
\subsection{The D-T Reaction and the Tritium Fuel Imperative}
\subsection{Breeder Blanket: Concept, Function, and Evolution}
\subsection{The Role of Neutron Multipliers}
\subsection{Major Breeder Blanket Architectures (Bullet Point to Subsection Demo)}

Blanket designs are broadly divided into two categories—liquid and solid—based on the phase of the
lithium-bearing breeder material.

\subsubsection{Liquid Breeder Concepts}

Liquid breeders (LBs) are attractive because the breeding medium can simultaneously function as the
coolant, simplifying the design and allowing for continuous tritium extraction outside the reactor
\cite{tas_liquid_1988}.

\paragraph{Lithium-Lead (\ce{Li-Pb}):}
    This is one of the most mature concepts, typically using the eutectic alloy \ce{Li17Pb83}
    \cite{mas_de_les_valls_leadlithium_2008, wu_overview_2011}. The lead acts as both a neutron
    multiplier and the primary component of the alloy, while the eutectic composition provides a low
    melting point, which is crucial for circulation \cite{mas_de_les_valls_leadlithium_2008}. This
    concept is the basis for the \textbf{Dual-Cooled Lithium-Lead (DCLL)} blanket, a primary
    candidate for the European DEMO reactor \cite{fernandez-berceruelo_progress_2024,
    malang_self-cooled_1991}. A key challenge for all liquid metal blankets is magnetohydrodynamics
    (MHD), where the strong magnetic fields of the tokamak induce currents in the flowing metal,
    creating a drag force that inhibits circulation \cite{malang_self-cooled_1991}.

\paragraph{Molten Salts:}
    An alternative LB concept uses molten fluorine or chlorine salts. The most prominent example is
    \textbf{FLiBe \ce{(2LiF . BeF2)}}, which advantageously combines the breeder (LiF) and
    multiplier (\ce{BeF2}) into a single, low-conductivity fluid \cite{tas_liquid_1988}. This is the
    reference design for the \textbf{Affordable Robust Compact (ARC)} reactor concept from MIT
    \cite{sorbom_arc_2015, bocci_arc_2020, segantin_optimization_2020}. A wide variety of other salt
    compositions, including \ce{LiF.PbF2} and novel chlorine-based salts, are also under
    investigation to optimise breeding, temperature, and material compatibility
    \cite{segantin_neutronic_2020, bohm_initial_2023}.

\subsubsection{Solid Breeder Concepts}


\section{Methodology}



\section{Results}


% revtex class:
% \bibliographystyle{plainnat}
% \bibliography{thesis}

% base article class:
\printbibliography  % for biblatex.


% --- END OF SECTION ---



\end{document}
%
% ****** End of file apssamp.tex ******

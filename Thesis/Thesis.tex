% DOC SETUP

% revtex class:
% \documentclass[%
%  onecolumn,
%  nofootinbib,
%  amsmath,amssymb,
%  aps,
%  12pt,
%  nolongbibliography,
% ]{revtex4-2}

% base article class:
\documentclass[12pt]{article}
% revtex loaded these for you, so now we add them explicitly
\usepackage{amsmath,amssymb}

% base article class:
% % Add the biblatex package
\usepackage[
  backend=biber,
  style=vancouver,  % Use the Vancouver style
  natbib=true,
  maxbibnames=6,    % Truncate if > 6 authors
  minbibnames=6     % <-- THE FIX: When truncating, print 6
]{biblatex}
% % Tell biblatex where to find your references
\addbibresource{thesis.bib} % <-- Make sure to include the .bib extension
\renewcommand*{\bibfont}{\small\linespread{1.0}\selectfont}
\usepackage{graphicx}% Include figure files
\usepackage{dcolumn}% Align table columns on decimal point
\usepackage{bm}% bold math
\usepackage{hyperref}% add hypertext capabilities
\usepackage[title]{appendix}
\usepackage[a4paper, margin=2cm]{geometry}
\usepackage{titlesec}
% --- Custom Title Formatting and Spacing ---

% 1. Set numbering depth to include \paragraph (level 4)
\setcounter{secnumdepth}{4}

% 2. Define \section (Large, BOLD, UPPERCASE, Ragged-Right)
% We MUST use the un-starred version to use \MakeUppercase
\titleformat{\section}
  {\normalfont\large\bfseries\raggedright} % Format: Large, bold, no right-justification
  {\thesection.}                          % Label
  {1em}                                   % Space between label and title
  {\MakeUppercase}                        % Command to apply to the title text
\titlespacing*{\section}
  {0pt}{3.5ex plus 1ex minus .2ex}{2.3ex plus .2ex}

% 3. Define \subsection (Normal size, BOLD)
% This is now \normalsize and \bfseries (bold) as you requested
\titleformat{\subsection}
  {\normalfont\normalsize\bfseries} % <-- CHANGED: normalsize, bold
  {\thesubsection.}
  {1em}
  {}
\titlespacing*{\subsection}
  {0pt}{3.25ex plus 1ex minus .2ex}{1.5ex plus .2ex}
% 4. Define \subsubsection (Normal size, bold-italic)
% This is now \bfseries\itshape (bold-italic) to distinguish it.
\titleformat{\subsubsection}
  {\normalfont\normalsize\bfseries\itshape} % <-- CHANGED TO BOLD-ITALIC
  {\thesubsubsection.}
  {1em}
  {}
\titlespacing*{\subsubsection}
  {0pt}{2.5ex plus 0.5ex minus .2ex}{1.2ex plus .2ex}

% 5. Define \paragraph (Normal size, italic)
% This is now \itshape (italic) as requested.
\titleformat{\paragraph}
  {\normalfont\normalsize\itshape} % <-- CHANGED TO ITALIC
  {\theparagraph.}
  {1em}
  {}
\titlespacing*{\paragraph}
  {0pt}{1.5ex plus 0.2ex}{1ex}
% --- End of Custom Title Formatting ---
\usepackage{mhchem}
\usepackage{booktabs} % <-- Add this to your preamble if not already there


% revtex class:
% \usepackage{etoolbox} % Make sure this is in your preamble
% % Add commands before the environment starts
% \BeforeBeginEnvironment{thebibliography}{%
%%   \begingroup                 % Start a group to keep changes local
%   \linespread{1.0}\selectfont % Set line spacing to single
%   \small                      % Set font size to \small
% }

% % Add commands after the environment ends
% \AfterEndEnvironment{thebibliography}{%
%   \endgroup % End the local changes
% }




%\usepackage[showframe,%Uncomment any one of the following lines to test
%%scale=0.7, marginratio={1:1, 2:3}, ignoreall,% default settings
%%text={7in,10in},centering,
%%margin=1.5in,
%%total={6.5in,8.75in}, top=1.2in, left=0.9in, includefoot,
%%height=10in,a5paper,hmargin={3cm,0.8in},
%]{geometry}


%%%  Some modifications to reduce the space used by RevTex for sections, and use traditional numbering
% % Usual (decimal) numbering
% \renewcommand{\thesection}{\arabic{section}}
% \renewcommand{\thesubsection}{\thesection.\arabic{subsection}}
% \renewcommand{\thesubsubsection}{\thesubsection.\arabic{subsubsection}}

% % Fix references
% \makeatletter
% \renewcommand{\p@subsection}{}
% \renewcommand{\p@subsubsection}{}
% \makeatother

%% Use this if you want times new roman which is slightly more compact.
%\usepackage{times}
\setlength{\parindent}{10pt}
\linespread{1.2}


% \titlespacing\section{0pt}{12pt plus 4pt minus 2pt}{6pt plus 2pt minus 2pt}
% \titlespacing\subsection{0pt}{5pt plus 4pt minus 2pt}{5pt plus 2pt minus 2pt}
% \titlespacing\subsubsection{0pt}{4pt plus 4pt minus 2pt}{4pt plus 2pt minus 2pt}

% \titleformat*{\section}{\raggedright\bf\MakeUppercase}
% \titleformat*{\subsection}{\itshape}


%% READ BEFORE USE IF HAVING CITATION TROUBLES The natlib package (used for citaitons here) uses
% \citep{} to change the citations to have square brackets over the author and year. Just using
% \cite{} will only bracket the year. For more harvard citation options visit this url:
% https://library-guides.imperial.ac.uk/c.php?g=719784&p=5224027




% DOC BEGIN

\begin{document}

% --- title page ---
\begin{titlepage}
    \centering % Center all content on the title page

    \vspace*{2cm} % Add some space from the top

    % --- Manually formatted title ---
    {\Large \bfseries Neutron Transport and Tritium Breeding Modelling in Nuclear Fusion Reactor Breeder Blankets \par}
    \vspace{0.5em}
    {\normalsize MPhys Project Report \par}

    \vspace{3em} % Add space after the title

    % --- Manually formatted author ---
    {\large Vadan Khan \par}
    \vspace{0.5em}
    {4th Year MPhys, Department of Physics and Astronomy, University of Manchester \par}

    \vspace{2em} % Add space after the author

    % --- Manually formatted date ---
    {\large \today \par}

    \vfill % Pushes the abstract to the lower half of the page

    % --- Abstract ---
    % Use a minipage to constrain the width of the abstract
    \begin{minipage}{0.8\textwidth}
    \begin{abstract}
    The pursuit of practical nuclear fusion is focused on the deuterium-tritium (D-T) fuel cycle,
    widely considered the most feasible pathway for near-term power generation.
    A critical challenge, however, is the reliance on tritium, a radioactive
    isotope with an extremely limited external supply, necessitating a self-sufficient fuel cycle.
    The D-T fusion reaction produces most of its energy as 14.1~MeV neutrons, which can in turn be
    used to regenerate tritium through neutron-lithium interactions. The reactor subsystem that
    performs this function—the breeder blanket—is therefore essential for the viability of the
    D-T fuel cycle. It must simultaneously extract heat, shield reactor components, and maintain
    tritium self-sufficiency. This project applies Monte Carlo neutron transport simulations
    using \texttt{OpenMC} to investigate the neutronic performance of advanced breeder
    blanket designs, including heterogeneous and functionally graded geometries. The study
    aims to quantify how variations in material composition and spatial structure influence
    the tritium breeding ratio (TBR) and neutron flux profiles, and explores the potential
    for automated optimisation of blanket parameters through computational frameworks.
    The outcomes will inform the feasibility and efficiency of next-generation blanket
    concepts relevant to future magnetic confinement fusion reactors such a DEMO and STEP.
    \end{abstract}
    \end{minipage}

    \vfill % Fills the remaining space at the bottom

\end{titlepage}
% --- title page ---

\section{Introduction}

The pursuit of practical nuclear fusion represents one of the most ambitious and potentially
transformative scientific endeavours of the twenty-first century. As global energy demand continues
to rise and the environmental impacts of fossil fuel use become ever clearer, fusion offers the
prospect of a near-limitless, low-carbon energy source. Fusion energy aims to replicate, on Earth,
the processes that power the Sun — combining light nuclei to release energy through mass-energy
conversion. Unlike conventional nuclear fission, fusion promises inherent safety, reduced long-lived
radioactive waste, and abundant fuel sources.

Among the various fusion reactions studied, the deuterium-tritium (D-T) reaction has emerged as the
most viable for near-term applications due to its relatively high cross-section at achievable plasma
temperatures and its favourable energy yield. However, the practical realisation of a D-T reactor
introduces challenges that extend far beyond plasma confinement and heating. Chief among these is
the supply of tritium, a radioactive isotope that does not occur naturally in significant quantities
and must be produced artificially.

This challenge has led to the development of the breeder blanket concept — an integrated structure
surrounding the plasma that captures energetic neutrons produced in fusion reactions and uses them
to generate tritium through interactions with lithium. In addition to tritium production, breeder
blankets also serve several critical roles: moderating neutron flux, capturing energy for conversion
to electricity, and protecting reactor components from radiation damage.

Understanding the behaviour of neutrons within the blanket region is therefore central to the
success of any D-T fusion system. Computational modelling of neutron transport, energy deposition,
and tritium breeding provides vital insights into how different blanket materials and geometries
perform under fusion-like conditions. These neutronics studies inform the design and optimisation of
test blanket modules and full-scale reactor concepts.

This project contributes to that effort by using advanced Monte Carlo simulations to model neutron
interactions within candidate breeder blanket materials. Through this, it aims to identify promising
compositions and geometries capable of sustaining tritium self-sufficiency in future fusion power
plants.

\section{Theory}

\subsection{Fundamentals of Nuclear Fusion}

Nuclear fusion is the process by which two light atomic nuclei combine, or ``fuse,'' to form a single
heavier nucleus. This process releases a tremendous amount of energy if the mass of the final
nucleus is less than the combined mass of the reactants, with the ``lost'' mass converted to energy
according to $E=mc^2$.

For terrestrial power generation, the most promising reaction is that between the two heavy isotopes
of hydrogen, deuterium (d) and tritium (t) \cite{paris_lost_2023, heckrotte_factors_1971}:
$$
\ce{d + t -> n + \alpha} + 17.6 MeV
$$
This D-T reaction is the primary focus of mainstream fusion research, as it has a fusion
cross-section significantly higher than other candidate reactions (like D-D or
D-${}^{3}$He) at the ``low'' plasma temperatures achievable with current magnetic confinement
technology (Fig.~\ref{fig:fusion_cross_sections}) \cite{heckrotte_factors_1971}.

\begin{figure}[htbp]
    \centering
    % You can adjust the width as needed for your document layout
    \includegraphics[width=0.7\textwidth]{figures/breeding_cs.png}
    \caption{A comparison of fusion reaction reactivities total cross section $\sigma$ as a function
    of ion temperature (keV) for the D-T, D-D, and \ce{D}-\ce{^3He} fuel cycles. The D-T reaction's
    cross section peaks at a lower temperature and is significantly larger than the alternatives,
    making it the most accessible reaction for current and near-term fusion devices.}
    \label{fig:fusion_cross_sections}
\end{figure}

To initiate fusion, the positively charged nuclei must be given enough kinetic energy to overcome
their mutual electrostatic repulsion, known as the Coulomb barrier. In Magnetic Confinement Fusion
(MCF) devices, such as the \textbf{tokamak} \cite{wesson}, this is achieved by heating the
D-T gas into a plasma state at temperatures exceeding 100 million Kelvin. The hot plasma is then
confined by powerful magnetic fields, preventing it from touching the reactor walls. The 17.6 MeV of
energy released is partitioned between the two products: the alpha particle ($\alpha$, a helium
nucleus) carries 3.5 MeV, while the neutron (n) carries 14.1 MeV \cite{heckrotte_factors_1971}.

This energy split is fundamental to the reactor's operation. The charged alpha particle remains
trapped by the magnetic fields, depositing its energy into the plasma and helping to sustain its
high temperature (a process known as ``plasma burning''). The neutron, being electrically neutral, is
un-bothered by the magnetic fields and escapes the plasma immediately, carrying 80\% of the fusion
energy with it \cite{heckrotte_factors_1971}.

\subsection{Neutron Transport and Interactions}

The 14.1 MeV neutron is the primary vehicle for both energy extraction and fuel production. Once it
leaves the plasma, it travels into the surrounding structures, chiefly the \textbf{breeder blanket}.
The study of its journey and interactions with the blanket materials is the domain of
\textbf{neutron transport} \cite{kuridan_neutron_2023}.

As the neutron moves through matter, it interacts with atomic nuclei via two main processes:
scattering and absorption. The probability of any specific interaction occurring is defined by the
material's \textbf{microscopic cross-section ($\sigma$)}, a value that is highly dependent on the energy
of the incident neutron \cite{kuridan_neutron_2023}. These cross-sections are meticulously measured
and compiled in comprehensive libraries such as the Evaluated Nuclear Data File (ENDF)
\cite{brown_endfb-viii0_2018}.

Key interactions within the blanket include:
\subsubsection{Scattering Interactions}
    The neutron collides with a nucleus and ``bounces'' off, transferring a portion of its kinetic
    energy to the nucleus. In \textbf{elastic scattering $\ce{(n,n^{\prime})}$}, kinetic energy is
    conserved. In \textbf{inelastic scattering $\ce{(n,n^{\prime})}$}, the neutron excites the
    nucleus, which then de-excites by emitting a gamma ray, resulting in a larger energy loss for
    the neutron. Both processes are crucial for \textbf{moderation}—slowing the fast 14.1 MeV
    neutrons down. This thermal energy, deposited in the blanket material, is what is ultimately
    extracted by a coolant to generate electricity.

\subsubsection{Absorption Interactions}
    The neutron is captured by a nucleus. This can be a \textbf{radiative capture \ce{(n, \gamma)}}
    event, where the nucleus emits a gamma ray. This is often a parasitic reaction, as it removes a
    neutron from the system that could have been used for breeding. Alternatively, the absorption
    can induce \textbf{charged particle emission}, such as \ce{(n,p)} or \ce{(n, \alpha)} reactions.
    This is the fundamental mechanism used for tritium breeding \cite{kuridan_neutron_2023}.

\subsection{Tritium Breeding and Neutron Multiplication}

The primary nuclear function of the blanket is to use the fusion neutrons to ``breed'' new tritium
fuel. This is accomplished by bombarding lithium isotopes with the neutrons. There are two reactions
that produce tritium:
\begin{equation}
    \ce{^{6}Li + n -> \alpha + t} + 4.8 MeV
    \label{eq:li6_reaction}
\end{equation}

\begin{equation}
    \ce{^{7}Li + n -> \alpha + t + n^{\prime}} - 2.5 MeV
    \label{eq:li7_reaction}
\end{equation}
The \ce{^{6}Li} reaction is exothermic and has a very large cross-section for low-energy (thermal)
neutrons, making it the primary breeding reaction of interest \cite{hernandez_first_2018}. The
\ce{^{7}Li} reaction is endothermic, requiring high-energy neutrons (a ``threshold'' reaction), and is
generally less effective.

\begin{figure}[htbp]
    \centering
    % Placeholder for the image as external files are not supported
    \includegraphics[width=0.7\textwidth]{figures/lithium_cs.png}
    \caption{Neutron cross-sections for the primary tritium breeding reactions,
    \ce{^{6}Li(n, \alpha)t} and $\ce{^{7}Li(n, \alpha t)n^{\prime}}$, as a function of incident neutron energy.
    The \ce{^{6}Li} cross-section is dominant at low (thermal) energies, while the \ce{^{7}Li}
    reaction is a threshold reaction requiring high-energy neutrons.}
    \label{fig:lithium_cross_sections}
\end{figure}

The critical metric for the blanket's success is the \textbf{Tritium Breeding Ratio (TBR)}, defined as
the ratio of tritium atoms produced in the blanket to the tritium atoms consumed in the plasma
\cite{federici_overview_2019}. Due to inevitable fuel cycle losses (e.g., incomplete plasma burn-up,
decay during extraction), the blanket must breed more tritium than is consumed. To achieve a
self-sufficient fuel cycle, a TBR $>$ 1.0 is mandatory, with most reactor designs targeting a value of
1.1 or higher \cite{sawan_physics_2006, meschini_modeling_2023}.

This presents a ``neutron economy'' challenge. The D-T reaction produces one neutron, and the primary
\ce{^{6}Li} breeding reaction consumes one neutron \cite{shatalov_breeder_1991}. This leaves no
margin for neutrons that are inevitably lost, either by leaking out of the blanket or through
parasitic absorption in structural materials (like steel) \cite{shatalov_breeder_1991}.

To overcome this deficit, the blanket must include a \textbf{neutron multiplier}
\cite{malang_self-cooled_1991, hernandez_first_2018}. These are materials that undergo an
\textbf{\ce{(n, 2n)}} reaction, where one high-energy incident neutron strikes a nucleus and causes two
neutrons to be emitted. The two most viable multiplier materials are Beryllium (Be) and Lead (Pb)
\cite{hernandez_first_2018}.
$$
\ce{^{9}Be + n -> ^{8}Be + 2n -> 2\alpha + 2n}
$$
$$
\ce{^{208}Pb + n -> ^{207m}Pb + 2n}
$$
By incorporating these materials, the single 14.1 MeV fusion neutron can be multiplied into two or
more lower-energy neutrons. These neutrons are then moderated (slowed down) within the blanket until
they are at the optimal thermal energy to be captured by \ce{^{6}Li}, thus enabling a TBR
significantly greater than one \cite{hernandez_first_2018}.

\begin{figure}[htbp]
    \centering
    % Placeholder for the image as external files are not supported
    \includegraphics[width=0.7\textwidth]{figures/multipliers_cs.png}
    \caption{Neutron cross-sections for the \ce{(n, 2n)} neutron multiplication reactions in
    Beryllium (\ce{^{9}Be}) and Lead (\ce{^{208}Pb}). Both are threshold reactions that require
    high-energy neutrons, such as those from the D-T fusion reaction (14.1 MeV), to effectively
    multiply the neutron population.}
    \label{fig:multiplier_cross_sections}
\end{figure}

\section{Literature Review}
\label{sec:lit_review}

\subsection{The D-T Reaction and the Tritium Fuel Imperative}

The deuterium-tritium (D-T) fusion reaction, \ce{d + t -> n + \alpha + 17.6 MeV}, has been the leading
candidate for first-generation fusion power plants since its potential was identified in the 1940s
\cite{paris_lost_2023}. Its primary advantage lies in its high fusion cross-section ($\sigma_f$) at
plasma temperatures achievable with current technology, significantly higher than other candidates
like D-D or \ce{D}-\ce{^3He} reactions. Furthermore, 80\% of the energy (14.1 MeV) is carried by the
emitted neutron, which allows for energy to be efficiently extracted from the plasma core
\cite{heckrotte_factors_1971}.

This reliance on the D-T reaction, however, creates a fundamental logistical challenge: the fuel
cycle is not self-contained. While deuterium is abundant in seawater, tritium is a radioactive
isotope with a short half-life of 12.32 years and is not found in nature in any significant
quantity. The world's current tritium supply is a by-product of heavy-water-moderated fission
reactors, such as CANDU reactors. Multiple analyses of this supply chain have concluded that global
stockpiles are extremely limited and face serious shortages \cite{kovari_tritium_2018,
pearson_tritium_2018}, \textit{even without} the projected demand from a future fleet of fusion
power plants.

Therefore, for fusion energy to be a sustainable and viable long-term power source, any commercial
D-T reactor must breed its own tritium. This has led to the development of in-situ fuel production
systems, known as \textbf{Tritium Breeder Blankets}.

\subsection{Breeder Blanket: Concept, Function, and Evolution}

A breeder blanket is a complex component surrounding the plasma-containing vacuum vessel. Its
primary function is to intercept the 14.1 MeV fusion neutrons and use them to induce
tritium-producing reactions in lithium. These are in reference to the two key breeding equations
Eq.~\eqref{eq:li6_reaction} and Eq.~\eqref{eq:li7_reaction}.

While \ce{^{7}Li} can breed, the \ce{^{6}Li} reaction is exothermic and possesses a much larger
reaction cross-section for the thermalised neutrons dominant in a blanket, making it the primary
breeding isotope.

The concept of a ``breeder'' blanket predates its application in fusion; it was first developed as a
method to use fusion neutrons to produce fissile materials, such as \ce{^{233}U} and \ce{^{239}Pu},
for fission reactors \cite{moir_fusion_1982, lee_us-doe_1983, maniscalco_fusion_1984}. Today, the
design must also perform several other critical functions, including shielding the superconducting
magnets from intense neutron radiation and extracting the fusion energy (via neutron thermalisation)
for power conversion.

The key metric for a blanket's performance is the \textbf{Tritium Breeding Ratio (TBR)}, defined as
the ratio of tritium atoms produced in the blanket to the tritium atoms consumed in the plasma. Due
to inevitable losses from neutron capture in structural materials, decay, and incomplete fuel
burn-up, a TBR significantly greater than 1.0 is required to achieve tritium self-sufficiency
\cite{sawan_physics_2006, meschini_modeling_2023}.

However, the D-T reaction produces only one neutron, and the \ce{^{6}Li} reaction consumes one
neutron. This leaves no margin for losses. To achieve a TBR $>$ 1, most blanket designs must include
a \textbf{neutron multiplier} material. The most effective materials for this are beryllium
(\ce{^{9}Be}) and lead (\ce{^{208}Pb}), which undergo \ce{(n, 2n)} reactions (spallation) when struck by
high-energy fusion neutrons, effectively turning one incident neutron into two
\cite{hernandez_first_2018}.

\subsection{The Role of Neutron Multipliers}

The D-T fusion reaction produces a single 14.1 MeV neutron, while the primary breeding reaction,
\ce{^{6}Li(n, \alpha)t}, consumes one neutron. This 1-to-1 neutron economy leaves no margin for
inevitable losses due to parasitic absorption in structural materials or neutrons escaping the
blanket. To achieve the required Tritium Breeding Ratio (TBR) greater than unity, a blanket must
therefore incorporate a \textbf{neutron multiplier} \cite{malang_self-cooled_1991}.

Natural lithium itself has a minor neutron-multiplying reaction via the high-energy
$\ce{(n,n^{\prime}t)}$ reaction with \ce{^{7}Li}, shown in Eq.~\eqref{eq:li7_reaction}. However this
reaction is endothermic, consuming 2.5 MeV of energy, and its cross-section is insufficient on its
own to guarantee tritium self-sufficiency \cite{hernandez_first_2018}. This leads to a common
consensus that dedicated neutron multipliers are required with the two primary candidates being
Beryllium (\ce{Be}) and Lead (\ce{Pb}).

\subsubsection{Beryllium Neutron Multipliers}
    Beryllium, specifically the isotope \ce{^{9}Be}, is an extremely effective neutron multiplier
    with a low energy threshold for its \ce{(n, 2n)} reaction:
    $$\ce{^{9}Be + n -> ^{8}Be + 2n -> 2\alpha} + \ce{2n}$$ It is widely used in conceptual solid breeder
    designs, such as the Helium-Cooled Pebble Bed (HCPB) and Water-Cooled Pebble Bed (WCPB), often
    in the form of beryllium pebbles or beryllide ceramics (\ce{Be12Ti}) \cite{wang_current_2019,
    liu_design_2022}. It is also a key component in the molten salt FLiBe \ce{(2LiF . BeF2)}
    \cite{tas_liquid_1988, sorbom_arc_2015}. However, beryllium has two major drawbacks: it is a
    scarce resource, and its dust is ``ridiculously toxic'' \cite{shimwell_reducing_2016}, posing
    significant safety, handling, and manufacturing challenges.

\subsubsection{Lead Neutron Multipliers}
    Lead, typically natural lead or lead enriched in \ce{^{208}Pb}, is the other main multiplier
    candidate. It undergoes a high-energy \ce{(n, 2n)} reaction:
    $$\ce{^{208}Pb + n -> ^{207m}Pb + 2n}$$ Lead's primary advantage is that it can be seamlessly
    integrated into a liquid breeder, forming the \ce{Li17Pb83} eutectic alloy. This allows the
    material to serve as breeder, multiplier, and coolant simultaneously \cite{wu_overview_2011,
    malang_self-cooled_1991}. While it avoids the toxicity issues of beryllium, lead is a high-Z
    (high atomic number) material. This means it can also be a source of parasitic neutron
    absorption, particularly for the low-energy thermal neutrons that are most effective for
    breeding with \ce{^{6}Li}. Furthermore, its activation by high-energy neutrons can lead to the
    production of long-lived radioisotopes, complicating waste disposal \cite{bocci_arc_2020}.

The choice of multiplier, therefore, represents a critical trade-off between neutronic efficiency,
material toxicity, resource availability, and long-term waste management \cite{hernandez_first_2018,
shimwell_reducing_2016}.

\subsection{Major Breeder Blanket Architectures}

Blanket designs are broadly divided into two categories—liquid and solid—based on the phase of the
lithium-bearing breeder material.

\subsubsection{Liquid Breeder Concepts}

Liquid breeders (LBs) are attractive because the breeding medium can simultaneously function as the
coolant, simplifying the design and allowing for continuous tritium extraction outside the reactor
\cite{tas_liquid_1988}.

\paragraph{Lithium-Lead (\ce{Li-Pb}):}
    This is one of the most mature concepts, typically using the eutectic alloy \ce{Li17Pb83}
    \cite{mas_de_les_valls_leadlithium_2008, wu_overview_2011}. The lead acts as both a neutron
    multiplier and the primary component of the alloy, while the eutectic composition provides a low
    melting point, which is crucial for circulation \cite{mas_de_les_valls_leadlithium_2008}. This
    concept is the basis for the \textbf{Dual-Cooled Lithium-Lead (DCLL)} blanket, a primary
    candidate for the European DEMO reactor \cite{fernandez-berceruelo_progress_2024,
    malang_self-cooled_1991}. A key challenge for all liquid metal blankets is magnetohydrodynamics
    (MHD), where the strong magnetic fields of the tokamak induce currents in the flowing metal,
    creating a drag force that inhibits circulation \cite{malang_self-cooled_1991}.

\paragraph{Molten Salts:}
    An alternative LB concept uses molten fluorine or chlorine salts. The most prominent example is
    \textbf{FLiBe \ce{(2LiF . BeF2)}}, which advantageously combines the breeder (LiF) and
    multiplier (\ce{BeF2}) into a single, low-conductivity fluid \cite{tas_liquid_1988}. This is the
    reference design for the \textbf{Affordable Robust Compact (ARC)} reactor concept from MIT
    \cite{sorbom_arc_2015, bocci_arc_2020, segantin_optimization_2020}. A wide variety of other salt
    compositions, including \ce{LiF.PbF2} and novel chlorine-based salts, are also under
    investigation to optimise breeding, temperature, and material compatibility
    \cite{segantin_neutronic_2020, bohm_initial_2023}.

\subsubsection{Solid Breeder Concepts}

Solid breeders (SBs) use lithium-based ceramics, typically in the form of packed pebble beds, which
eliminates MHD issues and offers high chemical stability \cite{proust_solid_1991,
knitter_recent_2013}.

\paragraph{Materials and Design:}
    The leading candidate materials are lithium metatitanate
    (\ce{Li2TiO3}) and lithium orthosilicate (\ce{Li4SiO4}), often mixed with separate beryllium-based
    pebbles (e.g., \ce{Be12Ti}) as a multiplier \cite{hernandez_first_2018, hernandez_chapter_2025}.
    A review by \cite{hernandez_first_2018} provides a comprehensive survey of these and other
    solid-phase options.

\paragraph{Implementations:}
    These materials form the basis of several major international
    designs. The \textbf{Chinese Fusion Engineering Test Reactor (CFETR)} is developing a
    \textbf{Water-Cooled Pebble Bed (WCPB)} blanket using \ce{Li2TiO3}/\ce{Be12Ti} pellets
    \cite{liu_design_2022, lei_design_2020}. European and Japanese DEMO concepts have focused on
    \textbf{Helium-Cooled Pebble Bed (HCPB)} designs, which are also being developed for ITER Test
    Blanket Modules \cite{federici_european_2017, wang_current_2019, kawamura_overview_2024}.

The primary challenges for SB concepts are the complex engineering required for cooling (either with
high-pressure helium or water) and the difficulty of extracting tritium, which must diffuse out of
the solid ceramic and can be trapped in radiation-induced vacancies \cite{shi_density_2015}.

\subsection{Neutronics Modelling and the Research Gap}
\label{sec:research_gap}

Evaluating the TBR and other neutronic parameters of these complex designs is extremely difficult to
do experimentally \cite{delaporte-mathurin_advancing_2025, hirose_functional_2024}. While
small-scale mockups provide crucial data on material properties and tritium release, they cannot
replicate the full neutron spectrum and geometry of a power plant
\cite{delaporte-mathurin_advancing_2025}. Consequently, the field relies heavily on computational
modelling, particularly \textbf{Monte Carlo (MC) simulations}. The probabilistic nature of neutron
transport—defined by scattering, absorption, and multiplication cross-sections—is ideally suited to
the statistical sampling methods of MC codes \cite{kalos_monte_2008, kuridan_neutron_2023}.

Modern neutronic toolchains are central to this research effort. The field increasingly utilizes
state-of-the-art, open-source Monte Carlo codes, which have been validated against other established
codes and shown to accurately reproduce TBR and neutron flux calculations for fusion systems
\cite{romano_openmc_2015, bae_arc_2022, fradera_pre-conceptual_2021, king_high_2022}. A significant
hurdle in this process is the management of complex reactor geometries. Accurately translating
detailed Computer-Aided Design (CAD) models into a format usable by the neutronics code is a
well-known challenge, often requiring significant manual effort \cite{shimwell_paramak_2021} and
geometric simplification \cite{segantin_neutronic_2020}.

A review of the literature reveals that decades of research have narrowed the focus to a small
selection of ``safe'' materials (e.g., Li-Pb, \ce{Li4SiO4}, \ce{Li2TiO3}) chosen not only for
breeding but for their structural, chemical, and cost properties \cite{shatalov_breeder_1991,
federici_overview_2019}. This necessary pragmatism has left many other material combinations,
particularly those in complex molten salt systems, comparatively unexplored
\cite{hernandez_first_2018}.

However, this material-focused pragmatism is not the only factor limiting the current design space.
A significant methodological gap exists. Past neutronic studies have typically been forced to choose
between breadth and depth. On one hand, broad surveys like \cite{hernandez_first_2018} provide an
excellent ``first principles'' review of a wide range of breeder and multiplier materials. But to
compare so many options, these studies must rely on simplified, often one-dimensional, models, which
cannot capture the complex three-dimensional neutron transport effects within a realistic tokamak
geometry.

On the other hand, studies focused on specific reactor concepts, such as the molten salt analysis
for DEMO by \cite{boullon_molten_2021} or the liquid breeder comparison for ARC by
\cite{segantin_neutronic_2020}, provide vital, high-fidelity 3D analysis. However, their conclusions
are inherently tied to the \textit{fixed geometry} of that specific reactor design. These studies
typically compare a handful of materials within a single ``dedicated model,'' fixing crucial
parameters like blanket thickness and structural fractions. Consequently, they can answer ``What is
the best material for \textit{this} design?'' but not ``What is the best \textit{combination} of
material and geometry?'' For example, such studies often present TBR as a function of \ce{^{6}Li}
enrichment, but do not explore how varying the blanket thickness or material composition ratios
would fundamentally change those results.

This limited scope was, until recently, a computational necessity. Manually creating and simulating
thousands of unique, complex 3D CAD models for a full parametric sweep was computationally
intractable. Recent advances in scriptable, parametric modelling workflows have fundamentally
lowered this barrier, making it feasible to automatically generate and distribute simulations for
vast libraries of reactor designs. Work by \cite{goel_optimization_2023} has demonstrated the power
of this new toolchain by modelling \textit{several} distinct geometries. While this provides an
excellent proof-of-concept, but as an unpublished paper this foundational work represents an initial
step and leaves a significant opportunity for a more comprehensive and systematic exploration,
particularly regarding material composition.

This project aims to fill this gap. By leveraging an integrated modelling workflow, we will conduct
a broad, systematic investigation of TBR performance. This work builds upon previous studies by
moving beyond the 1D-vs-3D or material-vs-geometry trade-off. We will explore the \textit{coupled
design space} by systematically and simultaneously varying key parameters that are often fixed:
\ce{^{6}Li} enrichment, blanket thickness, and material composition ratios in a realistic 3D
geometry. This systematic parameter sweep will provide a comprehensive dataset to identify optimal
\textit{combinations} of parameters, identifying design points that previous, more narrowly-focused
studies may have overlooked.


\noindent\rule{\linewidth}{0.4pt}

\section{Methodology}

\subsection{Modelling Neutron Transport: From Boltzmann to Monte Carlo}

The physics of neutron interactions described in Section~\ref{sec:research_gap}---including
scattering, absorption, and multiplication---must be accurately modelled to determine the neutronic
performance of a breeder blanket. The behaviour of the entire neutron population (or \textbf{neutron
flux}) within the reactor geometry is formally governed by the \textbf{Boltzmann Transport Equation
(BTE)}.

The BTE is fundamentally a particle balance equation. In conceptual terms, it states that the rate
of change of neutrons in a specific volume of space, travelling in a specific direction with a
specific energy, is equal to the sum of all neutron gains and losses. Gains come from neutrons
scattering \textit{into} that state from other states and from external sources (like the D-T
plasma). Losses come from neutrons scattering \textit{out} of that state, being absorbed (either
parasitically or in a breeding reaction), or simply leaking out of the volume.

While comprehensive, the BTE is an complex integro-differential equation that is impossible to solve
analytically for a three-dimensional, heterogeneous geometry like a fusion reactor. Therefore, a
numerical approach is required. This project employs the \textbf{Monte Carlo (MC) method}, a
stochastic technique that has become a cornerstone of neutron transport modelling since its
development for neutronic simulations \cite{summerscales_hitting_2023}.

Instead of solving the BTE for the entire population at once, the MC method simulates the individual
life histories of millions, or billions, of discrete particles \cite{kalos_monte_2008}. In this
work, each ``particle'' is a neutron. A simulation proceeds as follows:
\begin{enumerate}
    \item A source neutron is ``born'' at the plasma, with an energy of 14.1~MeV and a random
    direction.
    \item The code uses the material cross-section data \cite{brown_endfb-viii0_2018} as a set of
    probability distributions to determine how far the neutron travels before its next interaction.
    \item At the interaction site, the code again uses the probabilistic data to decide
    \textit{what} happens: it may scatter (elastically or inelastically), be absorbed (e.g., by
    \ce{^{6}Li}), or cause a multiplication event (e.g., an \ce{(n,2n)} reaction in Beryllium or
    Lead).
    \item The neutron's energy and direction are updated, and if it still exists (i.e., was not
    absorbed), the process repeats. If it was a multiplication event, a new, secondary neutron is
    added to the simulation to be tracked.
    \item This "life history" continues until the neutron (and all its children) are
    "killed"---either by being absorbed or by escaping the model geometry.
\end{enumerate}

By simulating a sufficiently large number of these individual histories, the averaged behaviour of
all particles converges to a statistically robust solution of the BTE. This statistical approach
allows for the estimation of key neutronic parameters, such as the spatial distribution of the
neutron flux and, most critically for this project, the \textbf{Tritium Breeding Ratio (TBR)}.

\subsection{Computational Workflow and Toolchain}

As established in the \nameref{sec:lit_review} (Section~\ref{sec:lit_review}), modern neutronics
research relies on a toolchain of simulation and geometry codes. This project leverages this
approach by using two key open-source tools: \textbf{\texttt{OpenMC}} and \textbf{Paramak}.

\begin{itemize}
    \item \textbf{\texttt{OpenMC}:} All simulations in this project are performed using
    \textbf{\texttt{OpenMC}}, a state-of-the-art, open-source Monte Carlo transport code
    \cite{romano_openmc_2015}. \texttt{OpenMC} is widely used in the fusion research community and
    has been extensively validated against other codes for complex fusion neutronics applications,
    demonstrating high accuracy in calculating parameters like TBR and volumetric heating
    \cite{bae_arc_2022}.

    \item \textbf{Paramak:} To manage the complex reactor geometry, this project also uses
    \textbf{Paramak}. It is a Python-based CAD framework that automates the construction of
    parametric tokamak models compatible with \texttt{OpenMC} \cite{shimwell_paramak_2021}, and is a
    key contributer to this project in providing a fast paramaterised easy to access API to generate
    sweep of geometry features and a variety of fusion reactor designs.
\end{itemize}

\subsection{Simulation Strategy and Parametric Study}

To systematically explore the vast design space, this project implements a custom, automated
simulation pipeline. This pipeline leverages the modular, scriptable nature of \texttt{Paramak} and
\texttt{OpenMC}, enabling key geometry and material parameters to be varied programmatically. By
structuring the simulation inputs as single functional arguments, the pipeline is explicitly
designed for iterative loops, allowing for large-scale parametric sweeps.

The core workflow for each individual simulation follows a precise sequence:
\begin{enumerate}
    \item \textbf{Geometry Generation:} \texttt{Paramak} generates a 3D reactor model (which uses a
    \texttt{CadQuery} backend \cite{shimwell_paramak_2021}) based on a given set of input parameters
    (e.g., blanket thickness, component radii).
    \item \textbf{Geometry Conversion:} This CAD model is converted into a DAGMC (Direct Accelerated
    Geometry Monte Carlo) unstructured mesh geometry, a format optimized for particle transport
    codes.
    \item \textbf{Model Setup:} The DAGMC geometry is imported into \texttt{OpenMC}, where it is
    populated with the specified material definitions and assigned the D-T fusion neutron source
    distribution (14.1 MeV).
\end{enumerate}

The overall research strategy is executed in two main phases. First, the pipeline's validity is
established by modelling a basic, reference tokamak geometry. This initial step serves to benchmark
the workflow, confirm physically sensible results, and establish initial neutronic trends.

Following this validation, a more comprehensive, multi-parameter study is performed. This second
phase executes coupled simulations to explore how changes in one parameter (e.g., \ce{^{6}Li}
enrichment) affect the optimal value of another (e.g., blanket thickness). The pipeline's
architecture is designed for distributability, allowing these computationally intensive simulations
to be run in parallel. This systematic sweep will be applied to three key reactor design archetypes,
enabling an ``agnostic" rating of certain features and the identification of robust design principles
that hold true across different blanket concepts.

\section{Results}
\begin{table}[htbp]
    \centering
    \caption{A summary of solid breeder materials, their properties, and calculated
    TBR. (Note: TBR values are placeholders to be populated.)}
    \label{tab:solid_breeders}
    \begin{tabular}{l D{.}{.}{2} D{.}{.}{2} c c}
        \toprule
        \textbf{Material} & \multicolumn{1}{c}{\textbf{Li (molar \%)}} & \multicolumn{1}{c}{\textbf{$\rho$ (g/cm$^3$)}} & \textbf{TBR} & \textbf{Reference} \\
        \midrule
        \ce{Li2O} & 66.66 & 2.10 & x & [34] \\
        \ce{Li4SiO4} & 44.44 & 2.40 & x & [25] \\
        \ce{Li2TiO3} & 33.33 & 3.43 & x & [25] \\
        \midrule % --- This was the dashed line in your image ---
        \ce{Li8PbO6} & 53.33 & 4.28 & x & [25] \\
        \ce{Li8SiO6} & 53.33 & 2.20 & x & [25] \\
        \ce{Li8CoO6} & 53.33 & 2.47 & x & [25] \\
        \ce{Li8GeO6} & 53.33 & 2.64 & x & [25] \\
        \ce{Li8ZrO6} & 53.33 & 2.98 & x & [25] \\
        \ce{Li8SnO6} & 53.33 & 3.41 & x & [25] \\
        \ce{Li8CeO6} & 53.33 & 3.25 & x & [25] \\
        \ce{Li6MnO4} & 54.55 & 2.50 & x & [25] \\
        \ce{Li6CoO4} & 54.55 & 2.77 & x & [25] \\
        \ce{Li6ZnO4} & 54.55 & 2.86 & x & [25] \\
        \ce{Li6Zr2O7} & 40.00 & 3.56 & x & [25] \\
        \ce{Li5AlO4} & 50.00 & 2.25 & x & [25] \\
        \ce{Li5FeO4} & 50.00 & 2.64 & x & [25] \\
        \ce{Li4TiO4} & 44.44 & 2.57 & x & [25] \\
        \ce{Li4GeO4} & 44.44 & 3.16 & x & [25] \\
        \ce{Li2SiO3} & 33.3 & 2.53 & x & [34] \\
        \ce{Li2MnO2} & 40.00 & 3.90 & x & [25] \\
        \bottomrule
    \end{tabular}
\end{table}

\begin{table}[htbp]
    \centering
    \caption{A summary of liquid breeder materials, their properties, and calculated
    TBR. (Note: TBR values are placeholders to be populated.)}
    \label{tab:liquid_breeders}
    \begin{tabular}{l l D{.}{.}{2} D{.}{.}{2} c c}
        \toprule
        \textbf{Material} & \textbf{Ratio (molar \%)} & \multicolumn{1}{c}{\textbf{Li (molar \%)}} & \multicolumn{1}{c}{\textbf{$\rho$ (g/cm$^3$)}} & \textbf{TBR} & \textbf{Reference} \\
        \midrule
        Li & - & 100 & 0.47 & x & [28] \\
        Li-Pb & 17:83 & 17.00 & 11.00 & x & [9] \\
        \ce{LiF-BeF2} & 2:1 & 28.57 & 2.04 & x & [29] \\
        \ce{LiF-BeF2}* & 1:1 & 20.00 & 2.06 & x & [29] \\
        \midrule % --- This was the dashed line in your image ---
        LiF & - & 50.00 & 2.64 & x & [25] \\
        \ce{Li3N} & - & 75.00 & 1.30 & x & [25] \\
        \ce{LiF-PbF2} & 2:3 & 15.38 & 3.55 & x & [29] \\
        \ce{LiF-BeF2-NaF} & 1:1:1 & 14.29 & 2.15 & x & [29] \\
        \ce{LiF-NaF-KF} & 46.5:11.5:42 & 20.17 & 2.02 & x & [28] \\
        \ce{LiF-LiBr-NaBr} & 20:73:7 & 46.50 & 3.16 & x & [28] \\
        \ce{LiF-LiBr-NaF} & 14:79:7 & 45.69 & 3.20 & x & [28] \\
        \ce{LiF-LiI2} & 83.5:16.5 & 50.00 & 3.68 & x & [28] \\
        \ce{LiF-NaF-ZrF4} & 55:22:23 & 20.45 & 2.72 & x & [28] \\
        \ce{LiCl-BeCl2} & 1:1 & 20.00 & - & x & [29] \\
        \ce{LiCl-PbCl2} & 1:2 & 12.50 & 4.50 & x & [29] \\
        \ce{LiCl-KCl} & 7:3 & 35.00 & 1.60 & x & [29] \\
        \ce{LiCl-KCl}* & 52:48 & 26.00 & 1.60 & x & [29] \\
        \ce{LiCl-NaCl} & 72:28 & 36.00 & 1.60 & x & [29] \\
        \bottomrule
    \end{tabular}
\end{table}


% revtex class:
% \bibliographystyle{plainnat}
% \bibliography{thesis}

% base article class:
\printbibliography  % for biblatex.


% --- END OF SECTION ---



\end{document}
%
% ****** End of file apssamp.tex ******

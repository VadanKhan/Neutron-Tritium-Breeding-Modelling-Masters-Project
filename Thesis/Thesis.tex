% DOC SETUP

% revtex class:
% \documentclass[%
%  onecolumn,
%  nofootinbib,
%  amsmath,amssymb,
%  aps,
%  12pt,
%  nolongbibliography,
% ]{revtex4-2}

% base article class:
\documentclass[12pt]{article}
% revtex loaded these for you, so now we add them explicitly
\usepackage{amsmath,amssymb}

% base article class:
% % Add the biblatex package
\usepackage[
  backend=biber,
  style=vancouver,  % Use the Vancouver style
  natbib=true,
  maxbibnames=6,    % Truncate if > 6 authors
  minbibnames=6     % <-- THE FIX: When truncating, print 6
]{biblatex}
% % Tell biblatex where to find your references
\addbibresource{thesis.bib} % <-- Make sure to include the .bib extension
\renewcommand*{\bibfont}{\small\linespread{1.0}\selectfont}
\usepackage{graphicx}% Include figure files
\usepackage{dcolumn}% Align table columns on decimal point
\usepackage{bm}% bold math
\usepackage{hyperref}% add hypertext capabilities
\usepackage[title]{appendix}
\usepackage[a4paper, margin=2cm]{geometry}
\usepackage{titlesec}
\usepackage{mhchem}

% revtex class:
% \usepackage{etoolbox} % Make sure this is in your preamble
% % Add commands before the environment starts
% \BeforeBeginEnvironment{thebibliography}{%
%%   \begingroup                 % Start a group to keep changes local
%   \linespread{1.0}\selectfont % Set line spacing to single
%   \small                      % Set font size to \small
% }

% % Add commands after the environment ends
% \AfterEndEnvironment{thebibliography}{%
%   \endgroup % End the local changes
% }




%\usepackage[showframe,%Uncomment any one of the following lines to test
%%scale=0.7, marginratio={1:1, 2:3}, ignoreall,% default settings
%%text={7in,10in},centering,
%%margin=1.5in,
%%total={6.5in,8.75in}, top=1.2in, left=0.9in, includefoot,
%%height=10in,a5paper,hmargin={3cm,0.8in},
%]{geometry}


%%%  Some modifications to reduce the space used by RevTex for sections, and use traditional numbering
% % Usual (decimal) numbering
% \renewcommand{\thesection}{\arabic{section}}
% \renewcommand{\thesubsection}{\thesection.\arabic{subsection}}
% \renewcommand{\thesubsubsection}{\thesubsection.\arabic{subsubsection}}

% % Fix references
% \makeatletter
% \renewcommand{\p@subsection}{}
% \renewcommand{\p@subsubsection}{}
% \makeatother

%% Use this if you want times new roman which is slightly more compact.
%\usepackage{times}
\setlength{\parindent}{10pt}
\linespread{1.1}

\usepackage{titlesec}
\titlespacing\section{0pt}{12pt plus 4pt minus 2pt}{6pt plus 2pt minus 2pt}
\titlespacing\subsection{0pt}{5pt plus 4pt minus 2pt}{5pt plus 2pt minus 2pt}
\titlespacing\subsubsection{0pt}{4pt plus 4pt minus 2pt}{4pt plus 2pt minus 2pt}

\titleformat*{\section}{\raggedright\bf\MakeUppercase}
\titleformat*{\subsection}{\itshape}


%% READ BEFORE USE IF HAVING CITATION TROUBLES The natlib package (used for citaitons here) uses
% \citep{} to change the citations to have square brackets over the author and year. Just using
% \cite{} will only bracket the year. For more harvard citation options visit this url:
% https://library-guides.imperial.ac.uk/c.php?g=719784&p=5224027




% DOC BEGIN

\begin{document}

\title{\Large Neutron Transport and Tritium Breeding Modelling in Nuclear Fusion Reactor Breeder
Blankets \\
\normalsize MPhys Project Report}

% base article class:
\author{Vadan Khan \\
4th Year MPhys, Department of Physics and Astronomy, University of Manchester}

% revtex class:
% \affiliation{4th Year MPhys, Department of Physics and Astronomy, University of Manchester}

\date{\today}

\begin{abstract}
The pursuit of practical nuclear fusion is focused on the deuterium-tritium (D-T) fuel cycle,
widely considered the most feasible pathway for near-term power generation.
A critical challenge, however, is the reliance on tritium, a radioactive
isotope with an extremely limited external supply, necessitating a self-sufficient fuel cycle.
The D-T fusion reaction produces most of its energy as 14.1~MeV neutrons, which can in turn be
used to regenerate tritium through neutron-lithium interactions. The reactor subsystem that
performs this function—the breeder blanket—is therefore essential for the viability of the
D-T fuel cycle. It must simultaneously extract heat, shield reactor components, and maintain
tritium self-sufficiency. This project applies Monte Carlo neutron transport simulations
using \texttt{OpenMC} to investigate the neutronic performance of advanced breeder
blanket designs, including heterogeneous and functionally graded geometries. The study
aims to quantify how variations in material composition and spatial structure influence
the tritium breeding ratio (TBR) and neutron flux profiles, and explores the potential
for automated optimisation of blanket parameters through computational frameworks.
The outcomes will inform the feasibility and efficiency of next-generation blanket
concepts relevant to future magnetic confinement fusion reactors such a DEMO and STEP.
\end{abstract}

\maketitle

\clearpage

\section{Introduction}

The pursuit of practical nuclear fusion represents one of the most ambitious and potentially
transformative scientific endeavours of the twenty-first century. As global energy demand continues
to rise and the environmental impacts of fossil fuel use become ever clearer, fusion offers the
prospect of a near-limitless, low-carbon energy source. Fusion energy aims to replicate, on Earth,
the processes that power the Sun — combining light nuclei to release energy through mass-energy
conversion. Unlike conventional nuclear fission, fusion promises inherent safety, reduced long-lived
radioactive waste, and abundant fuel sources.

Among the various fusion reactions studied, the deuterium-tritium (D-T) reaction has emerged as the
most viable for near-term applications due to its relatively high cross-section at achievable plasma
temperatures and its favourable energy yield. However, the practical realisation of a D-T reactor
introduces challenges that extend far beyond plasma confinement and heating. Chief among these is
the supply of tritium, a radioactive isotope that does not occur naturally in significant quantities
and must be produced artificially.

This challenge has led to the development of the breeder blanket concept — an integrated structure
surrounding the plasma that captures energetic neutrons produced in fusion reactions and uses them
to generate tritium through interactions with lithium. In addition to tritium production, breeder
blankets also serve several critical roles: moderating neutron flux, capturing energy for conversion
to electricity, and protecting reactor components from radiation damage.

Understanding the behaviour of neutrons within the blanket region is therefore central to the
success of any D-T fusion system. Computational modelling of neutron transport, energy deposition,
and tritium breeding provides vital insights into how different blanket materials and geometries
perform under fusion-like conditions. These neutronics studies inform the design and optimisation of
test blanket modules and full-scale reactor concepts.

This project contributes to that effort by using advanced Monte Carlo simulations to model neutron
interactions within candidate breeder blanket materials. Through this, it aims to identify promising
compositions and geometries capable of sustaining tritium self-sufficiency in future fusion power
plants.

\section{Theory}

\subsection{Fundamentals of Nuclear Fusion}

Nuclear fusion is the process by which two light atomic nuclei combine, or ``fuse,'' to form a single
heavier nucleus. This process releases a tremendous amount of energy if the mass of the final
nucleus is less than the combined mass of the reactants, with the ``lost'' mass converted to energy
according to $E=mc^2$.

For terrestrial power generation, the most promising reaction is that between the two heavy isotopes
of hydrogen, deuterium (d) and tritium (t) \cite{paris_lost_2023, heckrotte_factors_1971}:
$$
d+t\rightarrow n+\alpha+17.6MeV
$$
This D-T reaction is the primary focus of mainstream fusion research, as it has a fusion
cross-section ($<\sigma v>$, an averaged measure of reaction chance given temperature effects
\cite{nuclearphysics}) significantly higher than other candidate reactions (like D-D or
D-${}^{3}$He) at the ``low'' plasma temperatures achievable with current magnetic confinement
technology (Fig.~\ref{fig:fusion_cross_sections}) \cite{heckrotte_factors_1971}.

\begin{figure}[htbp]
    \centering
    % You can adjust the width as needed for your document layout
    \includegraphics[width=0.6\textwidth]{figures/DTbetter.png}
    \caption{A comparison of fusion reaction reactivities ($<\sigma v>$) as a function of ion
    temperature (keV) for the D-T, D-D, and D-${}^{3}$He fuel cycles. The D-T reaction's
    reactivity peaks at a lower temperature and is significantly larger than the alternatives,
    making it the most accessible reaction for current and near-term fusion devices. Adapted from
    \cite{jakobs_fusion_2016}.}
    \label{fig:fusion_cross_sections}
\end{figure}

To initiate fusion, the positively charged nuclei must be given enough kinetic energy to overcome
their mutual electrostatic repulsion, known as the Coulomb barrier. In Magnetic Confinement Fusion
(MCF) devices, such as the \textbf{tokamak} \cite{wesson}, this is achieved by heating the
D-T gas into a plasma state at temperatures exceeding 100 million Kelvin. The hot plasma is then
confined by powerful magnetic fields, preventing it from touching the reactor walls. The 17.6 MeV of
energy released is partitioned between the two products: the alpha particle ($\alpha$, a helium
nucleus) carries 3.5 MeV, while the neutron (n) carries 14.1 MeV \cite{heckrotte_factors_1971}.

This energy split is fundamental to the reactor's operation. The charged alpha particle remains
trapped by the magnetic fields, depositing its energy into the plasma and helping to sustain its
high temperature (a process known as ``plasma burning''). The neutron, being electrically neutral, is
un-bothered by the magnetic fields and escapes the plasma immediately, carrying 80% of the fusion
energy with it \cite{heckrotte_factors_1971}.

\subsection{Neutron Transport and Interactions}

The 14.1 MeV neutron is the primary vehicle for both energy extraction and fuel production. Once it
leaves the plasma, it travels into the surrounding structures, chiefly the 	extbf{breeder blanket}. The
study of its journey and interactions with the blanket materials is the domain of \textbf{neutron transport} \cite{kuridan_neutron_2023}.

As the neutron moves through matter, it interacts with atomic nuclei via two main processes:
scattering and absorption. The probability of any specific interaction occurring is defined by the
material's 	extbf{microscopic cross-section ($\sigma$)}, a value that is highly dependent on the energy
of the incident neutron \cite{kuridan_neutron_2023}. These cross-sections are meticulously measured
and compiled in comprehensive libraries such as the Evaluated Nuclear Data File (ENDF)
\cite{brown_endfb-viii0_2018}.

Key interactions within the blanket include:
\begin{itemize}
    \item \textbf{Scattering:} The neutron collides with a nucleus and ``bounces'' off, transferring
    a portion of its kinetic energy to the nucleus. In \textbf{elastic scattering (n,n)}, kinetic
    energy is conserved. In \textbf{inelastic scattering (n,n')}, the neutron excites the nucleus,
    which then de-excites by emitting a gamma ray, resulting in a larger energy loss for the
    neutron. Both processes are crucial for \textbf{moderation}—slowing the fast 14.1 MeV neutrons
    down. This thermal energy, deposited in the blanket material, is what is ultimately extracted by
    a coolant to generate electricity.

    \item \textbf{Absorption:} The neutron is captured by a nucleus. This can be a \textbf{radiative
    capture (n,$\gamma$)} event, where the nucleus emits a gamma ray. This is often a parasitic
    reaction, as it removes a neutron from the system that could have been used for breeding.
    Alternatively, the absorption can induce \textbf{charged particle emission}, such as (n,p) or
    (n,$\alpha$) reactions. This is the fundamental mechanism used for tritium breeding
    \cite{kuridan_neutron_2023}.
\end{itemize}

\subsection{Tritium Breeding and Neutron Multiplication}

The primary nuclear function of the blanket is to use the fusion neutrons to ``breed'' new tritium
fuel. This is accomplished by bombarding lithium isotopes with the neutrons. There are two reactions
that produce tritium:
$$
{}^{6}Li+n\rightarrow\alpha+t+4.8MeV
$$
$$
{}^{7}Li+n\rightarrow\alpha+t+n'-2.5MeV
$$
The ${}^{6}$Li reaction is exothermic and has a very large cross-section for low-energy (thermal)
neutrons, making it the primary breeding reaction of interest \cite{hernandez_first_2018}. The
${}^{7}$Li reaction is endothermic, requiring high-energy neutrons (a ``threshold'' reaction), and is
generally less effective.

The critical metric for the blanket's success is the \textbf{Tritium Breeding Ratio (TBR)}, defined as
the ratio of tritium atoms produced in the blanket to the tritium atoms consumed in the plasma
\cite{federici_overview_2019}. Due to inevitable fuel cycle losses (e.g., incomplete plasma burn-up,
decay during extraction), the blanket must breed more tritium than is consumed. To achieve a
self-sufficient fuel cycle, a TBR $>$ 1.0 is mandatory, with most reactor designs targeting a value of
1.1 or higher \cite{sawan_physics_2006, meschini_modeling_2023}.

This presents a ``neutron economy'' challenge. The D-T reaction produces one neutron, and the primary
${}^{6}$Li breeding reaction consumes one neutron \cite{shatalov_breeder_1991}. This leaves no
margin for neutrons that are inevitably lost, either by leaking out of the blanket or through
parasitic absorption in structural materials (like steel) \cite{shatalov_breeder_1991}.

To overcome this deficit, the blanket must include a \textbf{neutron multiplier}
\cite{malang_self-cooled_1991, hernandez_first_2018}. These are materials that undergo an
\textbf{(n,2n)} reaction, where one high-energy incident neutron strikes a nucleus and causes two
neutrons to be emitted. The two most viable multiplier materials are Beryllium (Be) and Lead (Pb)
\cite{hernandez_first_2018}.

$$
{}^{9}Be+n\rightarrow{}^{8}Be+2n \rightarrow 2\alpha + 2n
$$
$$
{}^{208}Pb+n\rightarrow{}^{207m}Pb+2n
$$
By incorporating these materials, the single 14.1 MeV fusion neutron can be multiplied into two or
more lower-energy neutrons. These neutrons are then moderated (slowed down) within the blanket until
they are at the optimal thermal energy to be captured by ${}^{6}$Li, thus enabling a TBR
significantly greater than one \cite{hernandez_first_2018}.

\section{Literature Review}

\subsection{The D-T Reaction and the Tritium Fuel Imperative}

The deuterium-tritium (D-T) fusion reaction, $d+t\rightarrow n+\alpha+17.6MeV$, has been the leading
candidate for first-generation fusion power plants since its potential was identified in the 1940s
\cite{paris_lost_2023}. Its primary advantage lies in its high fusion cross-section ($\sigma_f$) at
plasma temperatures achievable with current technology, significantly higher than other candidates
like D-D or D-${}^{3}$He reactions. Furthermore, 80% of the energy (14.1 MeV) is carried by the
emitted neutron, which allows for energy to be efficiently extracted from the plasma core
\cite{heckrotte_factors_1971}.

This reliance on the D-T reaction, however, creates a fundamental logistical challenge: the fuel
cycle is not self-contained. While deuterium is abundant in seawater, tritium is a radioactive
isotope with a short half-life of 12.32 years and is not found in nature in any significant
quantity. The world's current tritium supply is a by-product of heavy-water-moderated fission
reactors, such as CANDU reactors. Multiple analyses of this supply chain have concluded that global
stockpiles are extremely limited and face serious shortages \cite{kovari_tritium_2018,
pearson_tritium_2018}, \textit{even without} the projected demand from a future fleet of fusion
power plants.

Therefore, for fusion energy to be a sustainable and viable long-term power source, any commercial
D-T reactor must breed its own tritium. This has led to the development of in-situ fuel production
systems, known as \textbf{Tritium Breeder Blankets}.

\subsection{Breeder Blanket: Concept, Function, and Evolution}

A breeder blanket is a complex component surrounding the plasma-containing vacuum vessel. Its
primary function is to intercept the 14.1 MeV fusion neutrons and use them to induce
tritium-producing reactions in lithium. The two key breeding reactions are:

$${}^{6}Li+n\rightarrow\alpha+t+4.8MeV$$
$${}^{7}Li+n\rightarrow\alpha+t+n-2.5MeV$$

While ${}^{7}$Li can breed, the ${}^{6}$Li reaction is exothermic and possesses a much larger
reaction cross-section for the thermalised neutrons dominant in a blanket, making it the primary
breeding isotope.

The concept of a ``breeder'' blanket predates its application in fusion; it was first developed as a
method to use fusion neutrons to produce fissile materials, such as ${}^{233}$U and ${}^{239}$Pu,
for fission reactors \cite{moir_fusion_1982, lee_us-doe_1983, maniscalco_fusion_1984}. Today, the
design must also perform several other critical functions, including shielding the superconducting
magnets from intense neutron radiation and extracting the fusion energy (via neutron thermalisation)
for power conversion.

The key metric for a blanket's performance is the \textbf{Tritium Breeding Ratio (TBR)}, defined as
the ratio of tritium atoms produced in the blanket to the tritium atoms consumed in the plasma. Due
to inevitable losses from neutron capture in structural materials, decay, and incomplete fuel
burn-up, a TBR significantly greater than 1.0 is required to achieve tritium self-sufficiency
\cite{sawan_physics_2006, meschini_modeling_2023}.

However, the D-T reaction produces only one neutron, and the ${}^{6}$Li reaction consumes one
neutron. This leaves no margin for losses. To achieve a TBR $>$ 1, most blanket designs must include
a \textbf{neutron multiplier} material. The most effective materials for this are beryllium
(${}^{9}Be$) and lead (${}^{208}Pb$), which undergo (n, 2n) reactions (spallation) when struck by
high-energy fusion neutrons, effectively turning one incident neutron into two
\cite{hernandez_first_2018}.

\subsection{The Role of Neutron Multipliers}

The D-T fusion reaction produces a single 14.1 MeV neutron, while the primary breeding reaction,
${}^{6}Li(n,\alpha)t$, consumes one neutron. This 1-to-1 neutron economy leaves no margin for
inevitable losses due to parasitic absorption in structural materials or neutrons escaping the
blanket. To achieve the required Tritium Breeding Ratio (TBR) greater than unity, a blanket must
therefore incorporate a \textbf{neutron multiplier} \cite{malang_self-cooled_1991}.

While natural lithium itself has a minor neutron-multiplying reaction via the high-energy (n,n't)
reaction with ${}^{7}$Li:
$$ {}^{7}Li+n\rightarrow\alpha+t+n' - 2.5MeV $$ This reaction is endothermic, consuming 2.5 MeV of
energy, and its cross-section is insufficient on its own to guarantee tritium self-sufficiency
\cite{hernandez_first_2018}. Therefore, dedicated multiplier materials are required, with the two
primary candidates being Beryllium (Be) and Lead (Pb).

\begin{itemize}
    \item \textbf{Beryllium (Be):} Beryllium, specifically the isotope ${}^{9}Be$, is an extremely
    effective neutron multiplier with a low energy threshold for its (n,2n) reaction:
    $$ {}^{9}Be+n\rightarrow{}^{8}Be+2n \rightarrow 2\alpha + 2n $$ It is widely used in conceptual
    solid breeder designs, such as the Helium-Cooled Pebble Bed (HCPB) and Water-Cooled Pebble Bed
    (WCPB), often in the form of beryllium pebbles or beryllide ceramics ($Be_{12}Ti$)
    \cite{wang_current_2019, liu_design_2022}. It is also a key component in the molten salt FLiBe
    (2LiF - BeF$_2$) \cite{tas_liquid_1988, sorbom_arc_2015}. However, beryllium has two major
    drawbacks: it is a scarce resource, and its dust is ``ridiculously toxic''
    \cite{shimwell_reducing_2016}, posing significant safety, handling, and manufacturing
    challenges.

    \item \textbf{Lead (Pb):} Lead, typically natural lead or lead enriched in ${}^{208}Pb$, is the
    other main multiplier candidate. It undergoes a high-energy (n,2n) reaction:
    $$ {}^{208}Pb+n\rightarrow{}^{207m}Pb+2n $$ Lead's primary advantage is that it can be
    seamlessly integrated into a liquid breeder, forming the $Li_{17}Pb_{83}$ eutectic alloy. This
    allows the material to serve as breeder, multiplier, and coolant simultaneously
    \cite{wu_overview_2011, malang_self-cooled_1991}. While it avoids the toxicity issues of
    beryllium, lead is a high-Z (high atomic number) material. This means it can also be a source of
    parasitic neutron absorption, particularly for the low-energy thermal neutrons that are most
    effective for breeding with ${}^{6}Li$. Furthermore, its activation by high-energy neutrons can
    lead to the production of long-lived radioisotopes, complicating waste disposal
    \cite{bocci_arc_2020}.
\end{itemize}

The choice of multiplier, therefore, represents a critical trade-off between neutronic efficiency,
material toxicity, resource availability, and long-term waste management \cite{hernandez_first_2018,
shimwell_reducing_2016}.

\subsection{Major Breeder Blanket Architectures}

Blanket designs are broadly divided into two categories—liquid and solid—based on the phase of the
lithium-bearing breeder material.

\subsubsection{Liquid Breeder Concepts}

Liquid breeders (LBs) are attractive because the breeding medium can simultaneously function as the
coolant, simplifying the design and allowing for continuous tritium extraction outside the reactor
\cite{tas_liquid_1988}.

\begin{itemize}
    \item \textbf{Lithium-Lead (Li-Pb):} This is one of the most mature concepts, typically using
    the eutectic alloy $Li_{17}Pb_{83}$ \cite{mas_de_les_valls_leadlithium_2008, wu_overview_2011}.
    The lead acts as both a neutron multiplier and the primary component of the alloy, while the
    eutectic composition provides a low melting point, which is crucial for circulation
    \cite{mas_de_les_valls_leadlithium_2008}. This concept is the basis for the \textbf{Dual-Cooled
    Lithium-Lead (DCLL)} blanket, a primary candidate for the European DEMO reactor
    \cite{fernandez-berceruelo_progress_2024, malang_self-cooled_1991}. A key challenge for all
    liquid metal blankets is magnetohydrodynamics (MHD), where the strong magnetic fields of the
    tokamak induce currents in the flowing metal, creating a drag force that inhibits circulation
    \cite{malang_self-cooled_1991}.

    \item \textbf{Molten Salts:} An alternative LB concept uses molten fluorine or chlorine salts.
    The most prominent example is \textbf{FLiBe (2LiF - BeF$_2$)}, which advantageously combines the
    breeder (LiF) and multiplier (BeF$_2$) into a single, low-conductivity fluid
    \cite{tas_liquid_1988}. This is the reference design for the \textbf{Affordable Robust Compact
    (ARC)} reactor concept from MIT \cite{sorbom_arc_2015, bocci_arc_2020,
    segantin_optimization_2020}. A wide variety of other salt compositions, including LiF-PbF$_2$
    and novel chlorine-based salts, are also under investigation to optimise breeding, temperature,
    and material compatibility \cite{segantin_neutronic_2020, bohm_initial_2023}.
\end{itemize}

\subsubsection{Solid Breeder Concepts}

Solid breeders (SBs) use lithium-based ceramics, typically in the form of packed pebble beds, which
eliminates MHD issues and offers high chemical stability \cite{proust_solid_1991,
knitter_recent_2013}.

\begin{itemize}
    \item \textbf{Materials and Design:} The leading candidate materials are lithium metatitanate
    ($Li_2TiO_3$) and lithium orthosilicate ($Li_4SiO_4$), often mixed with separate beryllium-based
    pebbles (e.g., $Be_{12}Ti$) as a multiplier \cite{hernandez_first_2018, hernandez_chapter_2025}.
    A review by \cite{hernandez_first_2018} provides a comprehensive survey of these and other
    solid-phase options.

    \item \textbf{Implementations:} These materials form the basis of several major international
    designs. The \textbf{Chinese Fusion Engineering Test Reactor (CFETR)} is developing a
    \textbf{Water-Cooled Pebble Bed (WCPB)} blanket using $Li_2TiO_3/Be_{12}Ti$ pellets
    \cite{liu_design_2022, lei_design_2020}. European and Japanese DEMO concepts have focused on
    \textbf{Helium-Cooled Pebble Bed (HCPB)} designs, which are also being developed for ITER Test
    Blanket Modules \cite{federici_european_2017, wang_current_2019, kawamura_overview_2024}.
\end{itemize}

The primary challenges for SB concepts are the complex engineering required for cooling (either with
high-pressure helium or water) and the difficulty of extracting tritium, which must diffuse out of
the solid ceramic and can be trapped in radiation-induced vacancies \cite{shi_density_2015}.

\subsection{Neutronics Modelling and the Research Gap}

Evaluating the TBR and other neutronic parameters of these complex designs is extremely difficult to
do experimentally. While small-scale mockups provide crucial data on material properties and tritium
release \cite{delaporte-mathurin_advancing_2025, hirose_functional_2024}, they cannot replicate the
full neutron spectrum and geometry of a power plant \cite{delaporte-mathurin_advancing_2025}.

Consequently, the field relies heavily on computational modelling, particularly \textbf{Monte Carlo
(MC) simulations}. The probabilistic nature of neutron transport—defined by scattering, absorption,
and multiplication cross-sections—is ideally suited to the statistical sampling methods of MC codes
\cite{kalos_monte_2008, kuridan_neutron_2023}.

\begin{itemize}
    \item \textbf{OpenMC and Paramak:} This project uses \textbf{OpenMC}, a state-of-the-art,
    open-source MC code that has been validated against other established codes (like MCNP) and
    shown to accurately reproduce TBR and neutron flux calculations for fusion systems
    \cite{romano_openmc_2015, bae_arc_2022}. It is increasingly used in fusion research for its robust
    capabilities \cite{fradera_pre-conceptual_2021, king_high_2022}. To manage the complex reactor
    geometry, we use \textbf{Paramak}, a Python-based CAD framework that automates the construction
    of parametric tokamak models compatible with OpenMC \cite{shimwell_paramak_2021}.

    \item \textbf{The Research Gap and Project Motivation:} A review of the literature reveals that
    decades of research have narrowed the focus to a small selection of ``safe'' materials (e.g.,
    Li-Pb, $Li_4SiO_4$, $Li_2TiO_3$) chosen not only for breeding but for their structural,
    chemical, and cost properties \cite{shatalov_breeder_1991, federici_overview_2019}. This
    necessary pragmatism has left many other material combinations, particularly those in complex
    molten salt systems, comparatively unexplored.

    While systematic neutronic studies have been performed \cite{hernandez_first_2018,
    segantin_neutronic_2020, boullon_molten_2021}, they often use simplified 1D models or focus on a
    limited set of parameters. Recent work using OpenMC and Paramak \cite{goel_optimization_2023}
    has demonstrated the power of this new toolchain, but provides an opportunity for expansion.

    This project fills this gap. By leveraging the automated workflow of OpenMC and Paramak, we will
    conduct a broad, systematic investigation of TBR performance. This work builds upon previous
    studies by exploring a wider range of breeder materials (including less-common salts) and,
    crucially, by systematically varying key parameters such as \textbf{${}^{6}$Li enrichment},
    \textbf{blanket thickness}, and \textbf{material composition ratios} in a realistic 3D geometry.
    This systematic parameter sweep will provide a comprehensive dataset to identify optimal design
    points that previous, more narrowly-focused studies may have overlooked.
\end{itemize}

\noindent\rule{\linewidth}{0.4pt}


% revtex class:
% \bibliographystyle{plainnat}
% \bibliography{thesis}

% base article class:
\printbibliography  % for biblatex.

\end{document}
%
% ****** End of file apssamp.tex ******

% DOC SETUP

\documentclass[%
 onecolumn,
 nofootinbib,
 amsmath,amssymb,
 aps,
 10pt,
 nolongbibliography,
]{revtex4-2}

\usepackage{graphicx}% Include figure files
\usepackage{dcolumn}% Align table columns on decimal point
\usepackage{bm}% bold math
\usepackage{hyperref}% add hypertext capabilities
\usepackage[title]{appendix}
\usepackage[a4paper, margin=2cm]{geometry}
\usepackage{titlesec}
%\usepackage[showframe,%Uncomment any one of the following lines to test
%%scale=0.7, marginratio={1:1, 2:3}, ignoreall,% default settings
%%text={7in,10in},centering,
%%margin=1.5in,
%%total={6.5in,8.75in}, top=1.2in, left=0.9in, includefoot,
%%height=10in,a5paper,hmargin={3cm,0.8in},
%]{geometry}


%%%  Some modifications to reduce the space used by RevTex for sections, and use traditional numbering
% Usual (decimal) numbering
\renewcommand{\thesection}{\arabic{section}}
\renewcommand{\thesubsection}{\thesection.\arabic{subsection}}
\renewcommand{\thesubsubsection}{\thesubsection.\arabic{subsubsection}}

% Fix references
\makeatletter
\renewcommand{\p@subsection}{}
\renewcommand{\p@subsubsection}{}
\makeatother

%% Use this if you want times new roman which is slightly more compact.
%\usepackage{times}
\setlength{\parindent}{10pt}
\linespread{0.94}

\usepackage{titlesec}
\titlespacing\section{0pt}{12pt plus 4pt minus 2pt}{6pt plus 2pt minus 2pt}
\titlespacing\subsection{0pt}{5pt plus 4pt minus 2pt}{5pt plus 2pt minus 2pt}
\titlespacing\subsubsection{0pt}{4pt plus 4pt minus 2pt}{4pt plus 2pt minus 2pt}

\titleformat*{\section}{\raggedright\bf\MakeUppercase}
\titleformat*{\subsection}{\itshape}


%% READ BEFORE USE IF HAVING CITATION TROUBLES The natlib package (used for citaitons here) uses
% \citep{} to change the citations to have square brackets over the author and year. Just using
% \cite{} will only bracket the year. For more harvard citation options visit this url:
% https://library-guides.imperial.ac.uk/c.php?g=719784&p=5224027




% DOC BEGIN

\begin{document}

\title{\Large Neutron Transport and Tritium Breeding Modelling in Nuclear Fusion Reactor Breeder
Blankets \\
\normalsize MPhys Project Report}

\author{Vadan Khan}
\affiliation{4th Year MPhys, Department of Physics and Astronomy, University of Manchester}
\date{\today}

\begin{abstract}
The pursuit of practical nuclear fusion demands not only plasma confinement and heating but also the
establishment of a sustainable fuel cycle. The deuterium-tritium (D-T) fusion reaction produces most
of its energy as 14.1~MeV neutrons, which can in turn be used to regenerate tritium through
neutron-lithium interactions. The reactor subsystem that performs this function—the breeder
blanket—must simultaneously extract heat, shield reactor components, and maintain tritium
self-sufficiency. This project applies Monte Carlo neutron transport simulations using
\texttt{OpenMC} to investigate the neutronic performance of advanced breeder blanket designs,
including heterogeneous and functionally graded geometries. The study aims to quantify how
variations in material composition and spatial structure influence the tritium breeding ratio (TBR)
and neutron flux profiles, and explores the potential for automated optimisation of blanket
parameters through computational frameworks. The outcomes will inform the feasibility and efficiency
of next-generation blanket concepts relevant to future magnetic confinement fusion reactors such a
DEMO and STEP.
\end{abstract}

\maketitle

\clearpage

\section{Introduction}

The pursuit of practical nuclear fusion represents one of the most ambitious and potentially
transformative scientific endeavours of the twenty-first century. As global energy demand continues
to rise and the environmental impacts of fossil fuel use become ever clearer, fusion offers the
prospect of a near-limitless, low-carbon energy source. Fusion energy aims to replicate, on Earth,
the processes that power the Sun — combining light nuclei to release energy through mass-energy
conversion. Unlike conventional nuclear fission, fusion promises inherent safety, reduced long-lived
radioactive waste, and abundant fuel sources.

Among the various fusion reactions studied, the deuterium-tritium (D-T) reaction has emerged as the
most viable for near-term applications due to its relatively high cross-section at achievable plasma
temperatures and its favourable energy yield. However, the practical realisation of a D-T reactor
introduces challenges that extend far beyond plasma confinement and heating. Chief among these is
the supply of tritium, a radioactive isotope that does not occur naturally in significant quantities
and must be produced artificially.

This challenge has led to the development of the breeder blanket concept — an integrated structure
surrounding the plasma that captures energetic neutrons produced in fusion reactions and uses them
to generate tritium through interactions with lithium. In addition to tritium production, breeder
blankets also serve several critical roles: moderating neutron flux, capturing energy for conversion
to electricity, and protecting reactor components from radiation damage.

Understanding the behaviour of neutrons within the blanket region is therefore central to the
success of any D-T fusion system. Computational modelling of neutron transport, energy deposition,
and tritium breeding provides vital insights into how different blanket materials and geometries
perform under fusion-like conditions. These neutronics studies inform the design and optimisation of
test blanket modules and full-scale reactor concepts.

This project contributes to that effort by using advanced Monte Carlo simulations to model neutron
interactions within candidate breeder blanket materials. Through this, it aims to identify promising
compositions and geometries capable of sustaining tritium self-sufficiency in future fusion power
plants.

\section{Theory}

\subsection{Fundamentals of Nuclear Fusion}

Fusion is the process by which two light atomic nuclei combine to form a heavier nucleus, releasing
energy in accordance with the mass–energy relation. The most widely studied reaction for terrestrial
energy generation is the deuterium–tritium (D–T) reaction, in which the two isotopes of hydrogen
combine to produce a helium nucleus and a high-energy neutron. The neutron, carrying most of the
reaction energy, plays a central role in the broader fusion system.

Achieving fusion requires overcoming the Coulomb barrier between the positively charged nuclei. In
magnetic confinement fusion (MCF) devices such as tokamaks and stellarators, this is accomplished by
heating the plasma to temperatures exceeding 100 million kelvin, ensuring a sufficient number of
particle collisions occur with energies above the barrier. The resulting fusion power depends on the
plasma density, temperature, and confinement time, as described by the Lawson criterion.

\subsection{Neutron Production and Transport}

Each DT fusion reaction produces a 14.1~MeV neutron that escapes the plasma almost unimpeded. These
high-energy neutrons interact with surrounding materials via scattering and absorption processes,
depositing energy and inducing nuclear reactions. Because neutrons are uncharged, they cannot be
confined electromagnetically, making the understanding of their transport through matter a key
aspect of reactor design.

The neutron energy spectrum, moderation behaviour, and spatial flux distribution determine how
effectively a blanket converts neutron energy into heat and produces tritium. Cross-section data for
neutron interactions are typically energy-dependent and are tabulated in nuclear data libraries used
by simulation codes.

\subsection{Tritium Breeding and Neutron Multiplication}

The breeder blanket's principal nuclear function is to generate tritium through neutron–lithium
interactions. Two primary reactions contribute to tritium production:
the $^{6}\mathrm{Li}(n,\alpha)^{3}\mathrm{H}$ and $^{7}\mathrm{Li}(n,n\alpha)^{3}\mathrm{H}$ channels.
The first reaction has a high cross-section for thermal neutrons,
while the second requires higher-energy neutrons but allows the use of natural lithium without
isotope enrichment. Efficient tritium breeding requires a careful balance between neutron moderation
 and multiplication.

To maintain a self-sufficient fuel cycle, the system must achieve a tritium breeding ratio (TBR)
greater than unity — typically around 1.1–1.2 to account for losses and inefficiencies. Achieving
such values often necessitates incorporating neutron-multiplying materials such as beryllium or
lead, which release additional neutrons through $(n,2n)$ reactions, thereby improving the overall
neutron economy.

\subsection{Neutronics Modelling}

Because direct experimentation on full-scale fusion reactors remains infeasible, numerical modelling
is essential for assessing blanket performance. Monte Carlo transport methods simulate individual
neutron histories, statistically sampling reaction probabilities to estimate flux, absorption, and
secondary particle production. Codes such as MCNP, TRIPOLI, and OpenMC implement these methods using
detailed geometry and material data.

OpenMC, an open-source Monte Carlo code, provides a flexible environment for studying fusion systems
with complex geometries imported from computer-aided design (CAD) models. Its ability to model
spatially resolved neutron flux and isotope generation makes it particularly well suited to the
analysis of breeder blanket designs — forming the computational foundation of this project.

\section{Literature Review}

\subsection{The D-T Reaction and the Tritium Fuel Imperative}

The deuterium-tritium (D-T) fusion reaction, $d+t\rightarrow n+\alpha+17.6MeV$, has been the leading
candidate for first-generation fusion power plants since its potential was identified in the 1940s .
Its primary advantage lies in its high fusion cross-section ($\sigma_f$) at plasma temperatures
achievable with current technology, significantly higher than other candidates like D-D or
D-${}^{3}$He reactions. Furthermore, the reaction is highly aneutronic, with 80\% of the energy (14.1
MeV) carried by the emitted neutron, which allows for energy to be efficiently extracted from the
plasma core.

This reliance on the D-T reaction, however, creates a fundamental logistical challenge: the fuel
cycle is not self-contained. While deuterium is abundant in seawater, tritium is a radioactive
isotope with a short half-life of 12.32 years and is not found in nature in any significant
quantity. The world's current tritium supply is a by-product of heavy-water-moderated fission
reactors, such as CANDU reactors. Multiple analyses of this supply chain have concluded that global
stockpiles are extremely limited and face serious shortages, \textit{even without} the projected
demand from a future fleet of fusion power plants.

Therefore, for fusion energy to be a sustainable and viable long-term power source, any commercial
D-T reactor must breed its own tritium. This has led to the development of in-situ fuel production
systems, known as \textbf{Tritium Breeder Blankets}.

\subsection{Breeder Blanket: Concept, Function, and Evolution}

A breeder blanket is a complex component surrounding the plasma-containing vacuum vessel. Its
primary function is to intercept the 14.1 MeV fusion neutrons and use them to induce
tritium-producing reactions in lithium. The two key breeding reactions are:

\begin{itemize}
    \item ${}^{6}Li+n\rightarrow\alpha+t+4.8MeV$
    \item ${}^{7}Li+n\rightarrow\alpha+t+n-2.5MeV$
\end{itemize}

While ${}^{7}$Li can breed, the ${}^{6}$Li reaction is exothermic and possesses a much larger
reaction cross-section for the thermalised neutrons dominant in a blanket, making it the primary
breeding isotope.

The concept of a "breeder" blanket predates its application in fusion; it was first developed as a
method to use fusion neutrons to produce fissile materials, such as ${}^{233}$U and ${}^{239}$Pu,
for fission reactors. Today, the design must also perform several other critical functions,
including shielding the superconducting magnets from intense neutron radiation and extracting the
fusion energy (via neutron thermalisation) for power conversion.

The key metric for a blanket's performance is the \textbf{Tritium Breeding Ratio (TBR)}, defined as
the ratio of tritium atoms produced in the blanket to the tritium atoms consumed in the plasma. Due
to inevitable losses from neutron capture in structural materials, decay, and incomplete fuel
burn-up, a TBR significantly greater than 1.0 is required to achieve tritium self-sufficiency.

However, the D-T reaction produces only one neutron, and the ${}^{6}$Li reaction consumes one
neutron. This leaves no margin for losses. To achieve a TBR > 1, most blanket designs must include a
\textbf{neutron multiplier} material. The most effective materials for this are beryllium
(${}^{9}Be$) and lead (${}^{208}Pb$), which undergo (n, 2n) reactions (spallation) when struck by
high-energy fusion neutrons, effectively turning one incident neutron into two.

\subsection{Major Breeder Blanket Architectures}

Blanket designs are broadly divided into two categories—liquid and solid—based on the phase of the
lithium-bearing breeder material.

\subsubsection{Liquid Breeder Concepts}

Liquid breeders (LBs) are attractive because the breeding medium can simultaneously function as the
coolant, simplifying the design and allowing for continuous tritium extraction outside the reactor.

\begin{itemize}
    \item \textbf{Lithium-Lead (Li-Pb):} This is one of the most mature concepts, typically using
    the eutectic alloy $Li_{17}Pb_{83}$. The lead acts as both a neutron multiplier and the primary
    component of the alloy, while the eutectic composition provides a low melting point, which is
    crucial for circulation. This concept is the basis for the \textbf{Dual-Cooled Lithium-Lead
    (DCLL)} blanket, a primary candidate for the European DEMO reactor, which uses helium to cool
    the first wall and the liquid Li-Pb alloy as a self-cooling breeder. A key challenge for all
    liquid metal blankets is magnetohydrodynamics (MHD), where the strong magnetic fields of the
    tokamak induce currents in the flowing metal, creating a drag force that inhibits circulation.

    \item \textbf{Molten Salts:} An alternative LB concept uses molten fluorine or chlorine salts.
    The most prominent example is \textbf{FLiBe (2LiF - BeF$_2$)}, which advantageously combines the
    breeder (LiF) and multiplier (BeF$_2$) into a single, low-conductivity fluid. This is the
    reference design for the \textbf{Affordable Robust Compact (ARC)} reactor concept from MIT. A
    wide variety of other salt compositions, including LiF-PbF$_2$ and novel chlorine-based salts,
    are also under investigation to optimise breeding, temperature, and material compatibility.
\end{itemize}

\subsubsection{Solid Breeder Concepts}

Solid breeders (SBs) use lithium-based ceramics, typically in the form of packed pebble beds, which
eliminates MHD issues and offers high chemical stability.

\begin{itemize}
    \item \textbf{Materials and Design:} The leading candidate materials are lithium metatitanate
    ($Li_2TiO_3$) and lithium orthosilicate ($Li_4SiO_4$), often mixed with separate beryllium-based
    pebbles (e.g., $Be_{12}Ti$) as a multiplier. A review by Hernández and Pereslavtsev provides a
    comprehensive survey of these and other solid-phase options.

    \item \textbf{Implementations:} These materials form the basis of several major international
    designs. The \textbf{Chinese Fusion Engineering Test Reactor (CFETR)} is developing a
    \textbf{Water-Cooled Pebble Bed (WCPB)} blanket using $Li_2TiO_3/Be_{12}Ti$ pellets. European
    and Japanese DEMO concepts have focused on \textbf{Helium-Cooled Pebble Bed (HCPB)} designs,
    which are also being developed for ITER Test Blanket Modules.
\end{itemize}

The primary challenges for SB concepts are the complex engineering required for cooling (either with
high-pressure helium or water) and the difficulty of extracting tritium, which must diffuse out of
the solid ceramic and can be trapped in radiation-induced vacancies.

\subsection{Neutronics Modelling and the Research Gap}

Evaluating the TBR and other neutronic parameters of these complex designs is extremely difficult to
do experimentally. While small-scale mockups provide crucial data on material properties and tritium
release, they cannot replicate the full neutron spectrum and geometry of a power plant.

Consequently, the field relies heavily on computational modelling, particularly \textbf{Monte Carlo
(MC) simulations}. The probabilistic nature of neutron transport—defined by scattering, absorption,
and multiplication cross-sections—is ideally suited to the statistical sampling methods of MC codes.

\begin{itemize}
    \item \textbf{OpenMC and Paramak:} This project uses \textbf{OpenMC}, a state-of-the-art,
    open-source MC code that has been validated against other established codes (like MCNP) and
    shown to accurately reproduce TBR and neutron flux calculations for fusion systems. It is
    increasingly used in fusion research for its robust capabilities. To manage the complex reactor
    geometry, we use \textbf{Paramak}, a Python-based CAD framework that automates the construction
    of parametric tokamak models compatible with OpenMC.

    \item \textbf{The Research Gap and Project Motivation:} A review of the literature reveals that
    decades of research have narrowed the focus to a small selection of "safe" materials (e.g.,
    Li-Pb, $Li_4SiO_4$, $Li_2TiO_3$) chosen not only for breeding but for their structural,
    chemical, and cost properties. This necessary pragmatism has left many other material
    combinations, particularly those in complex molten salt systems, comparatively unexplored.

    While systematic neutronic studies have been performed, they often use simplified 1D models or
    focus on a limited set of parameters. Recent work using OpenMC and Paramak has demonstrated the
    power of this new toolchain, but provides an opportunity for expansion.

    This project fills this gap. By leveraging the automated workflow of OpenMC and Paramak, we will
    conduct a broad, systematic investigation of TBR performance. This work builds upon previous
    studies by exploring a wider range of breeder materials (including less-common salts) and,
    crucially, by systematically varying key parameters such as \textbf{${}^{6}$Li enrichment},
    \textbf{blanket thickness}, and \textbf{material composition ratios} in a realistic 3D geometry.
    This systematic parameter sweep will provide a comprehensive dataset to identify optimal design
    points that previous, more narrowly-focused studies may have overlooked.
\end{itemize}


\noindent\rule{\linewidth}{0.4pt}

\bibliographystyle{plainnat}
\bibliography{thesis}


\end{document}
%
% ****** End of file apssamp.tex ******
